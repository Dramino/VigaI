\documentclass[11pt,a4paper]{article}
\usepackage{fp}
\usepackage[utf8]{inputenc}
\usepackage[spanish]{babel}	
\usepackage{amsmath}
\spanishdecimal{.}
\begin{document}
\section{Propiedades de perfil W}
\newcommand\dW{22.86}      % Peralte del perfil W shape
\newcommand\bfW{21.03}     % Ancho del perfil W shape
\newcommand\twW{1.45}      % Espesor de alma del perfil W shape
\newcommand\tfW{2.37}      % Espesor de patín del perfil W shape
\newcommand\fy{3515}       % Resistencia del acero
\newcommand\Ey{2039000}    % Modulo de elasticidad del acero

\subsection {Área}

\begin{equation}
	A_s=2b_f t_f+(d-2t_f)t_w
\end{equation}

\FPeval{\asW}{round(2*\bfW*\tfW+(\dW-2*\tfW)*\twW,2)}
\begin{align*}
	A_s=2(\bfW)(\tfW)+[\dW-2(\tfW)](\twW)=\asW \text{ cm$^2$}
\end{align*}

\subsection{Peso}
\begin{equation}
	W_t=A_s\gamma_s=A_s(7,850 \text{ kg/m$^3$})
\end{equation}
\FPeval{\pesoW}{round(\asW*7850/10000,2)}
\begin{align*}
	W_t=\frac{\asW(7850)}{10000}=\pesoW \text{ kg/cm}
\end{align*}

\subsection{Momentos de inercia}
\begin{equation}
	I_{xx}=\frac{b_fd^3-(b_f-t_w)(d-2t_f)^3}{12}
\end{equation}
\FPeval{\ixxW}{round((\bfW*\dW^3-(\bfW-\twW)*(\dW-2*\tfW)^3)/12,2)}
\begin{align*}
	I_{xx}=\frac{(\bfW)(\dW)^3-(\bfW-\twW)\left[\dW-2(\tfW)\right]^3}{12}=\ixxW \text{ cm$^4$}
\end{align*}

\begin{equation}
	I_{yy}=\frac{2t_f b_f^3+(d-2t_f)(t_w)^3}{12}
\end{equation}
\FPeval{\iyyW}{round((2*\tfW*\bfW^3+(\dW-2*\tfW)*\twW^3)/12,2)}
\begin{align*}
	I_{yy}=\frac{(2)(\tfW)(\bfW)^3+[\dW-2(\tfW)](\twW)^3}{12}=\iyyW \text{ cm$^4$}
\end{align*}

\subsection{Módulo de sección elástico}

\begin{equation}
	S_{x}=2\frac{I_{xx}}{d}
\end{equation}
\FPeval{\sxW}{round(2*\ixxW/\dW,2)}
\begin{align*}
	S_{x}=2\frac{\ixxW}{\dW}=\sxW \text{ cm$^3$}
\end{align*}

\begin{equation}
	S_{y}=2\frac{I_{yy}}{b_f}
\end{equation}
\FPeval{\syW}{round(2*\iyyW/\bfW,2)}
\begin{align*}
	S_{x}=2\frac{\iyyW}{\bfW}=\syW \text{ cm$^3$}
\end{align*}

\subsection{Módulo de sección plástico}
\begin{equation}
	Z_{x}=b_f t_f(d-t_f)+\left(\frac{d}{2}-t_f\right)^2t_w
\end{equation}
\FPeval{\zxW}{round(\bfW*\tfW*(\dW-\tfW)+(\dW/2-\tfW)^2*\twW,2)}
\begin{align*}
	Z_{x}=(\bfW)(\tfW)(\dW-\tfW)+\left(\frac{\dW}{2}-\tfW\right)^2\twW=\zxW \text{ cm$^3$}
\end{align*}

\begin{equation}
	Z_{y}=\frac{b_f^2 t_f}{2}+(d-2t_f)\frac{t_w^2}{4} 
\end{equation}
\FPeval{\zyW}{round(\bfW^2*\tfW/2+(\dW-2*\tfW)*\twW^2/4,2)}
\begin{align*}
	Z_{y}=\frac{(\bfW)^2 (\tfW)}{2}+\left[\dW-2(\tfW)\right]\frac{(\twW)^2}{4}=\zyW \text{ cm$^3$}
\end{align*}

\subsection{Radio de giro}
\begin{equation}
	r_{x}=\sqrt{\frac{I_{xx}}{A_s}} 
\end{equation}
\FPeval{\rxW}{round((\ixxW/\asW)^0.5,2)}
\begin{align*}
	r_{x}=\sqrt{\frac{\ixxW}{\asW}}=\rxW \text{ cm}
\end{align*}

\begin{equation}
	r_{y}=\sqrt{\frac{I_{yy}}{A_s}} 
\end{equation}
\FPeval{\ryW}{round((\iyyW/\asW)^0.5,2)}
\begin{align*}
	r_{y}=\sqrt{\frac{\iyyW}{\asW}}=\ryW \text{ cm}
\end{align*}

\subsection{Momento resistente}
\subsubsection{ASD}
\begin{equation}
	M_{x}=0.9S_x f_y 
\end{equation}
\FPeval{\mxWasd}{round(0.9*\sxW*\fy/100000,2)}
\begin{align*}
	M_{x}=0.9 (\sxW) (\fy) \times10^{-5}=\mxWasd  \text{ ton-m}
\end{align*}

\begin{equation}
	M{y}=0.9S_y f_y 
\end{equation}
\FPeval{\myWasd}{round(0.9*\syW*\fy/100000,2)}
\begin{align*}
	M_{y}=0.9 (\syW) (\fy) \times10^{-5}=\myWasd  \text{ ton-m}
\end{align*}

\subsubsection{LRFD}
\begin{equation}
	M_{x}=0.9Z_x f_y 
\end{equation}
\FPeval{\mxWlrfd}{round(0.9*\zxW*\fy/100000,2)}
\begin{align*}
	M_{x}=0.9 (\zxW) (\fy) \times10^{-5}=\mxWlrfd  \text{ ton-m}
\end{align*}

\begin{equation}
	M_{y}=0.9Z_y f_y 
\end{equation}
\FPeval{\myWlrfd}{round(0.9*\zyW*\fy/100000,2)}
\begin{align*}
	M_{y}=0.9 (\zyW) (\fy) \times10^{-5}=\myWlrfd  \text{ ton-m}
\end{align*}

\subsubsection{Longitudes largas donde hay alabeo del patín}
Cuando la distancia es muy larga y no esta arriostrada la viga, el patín tiende a alabearse por lo que no alcanza su máxima capacidad. Existen dos distancias que índican los limites de longitud para este comportamiento que son $L_p$ y $L_r$

\begin{equation}
	L_p=1.76r_y\sqrt{\frac{E}{f_y}} 
\end{equation}
\FPeval{\lpW}{round(1.76*\ryW*(\Ey/\fy)^0.5,2)}
\begin{align*}
	L_p=1.76(\ryW)\sqrt{\frac{\Ey}{\fy}} =\lpW \text{ cm}
\end{align*}

Para obtener $L_r$ se requiere la constante de alabeo $C_w$, la constante torsional de St. Venant  $J$ y el valor de $r_{ts}$

\begin{equation}
	C_w=\frac{t_f(d-t_f)^2b_f^3}{24} 
\end{equation}
\FPeval{\cwW}{round((\tfW*(\dW-\tfW)^2*\bfW^3)/24,2)}
\begin{align*}
	C_w=\frac{\tfW(\dW-\tfW)^2(\bfW)^3}{24}  =\cwW \text{ cm$^6$}
\end{align*}

\begin{equation}
	r_{ts}=\sqrt{\frac{\sqrt{I_{yy}C_w}}{S_x}}
\end{equation}
\FPeval{\rtsW}{round(((\iyyW*\cwW)^0.5/\sxW)^0.5,2)}
\begin{align*}
	r_{ts}=\sqrt{\frac{\sqrt{(\iyyW)(\cwW)}}{\sxW}}= \rtsW\text{ cm}
\end{align*}

\begin{equation}
	J=\frac{1}{3}\left[2b_ft_{f}^{3}+(d-t_f)t_w^3 \right]
\end{equation}
\FPeval{\jW}{round((2*\bfW*\tfW^3+(\dW-\tfW)*\twW^3)/3,2)}
\begin{align*}
	J=\frac{1}{3}\left[2(\bfW)(\tfW)^{3}+(\dW-\tfW)(\twW)^3 \right]= \jW\text{ cm$^4$}
\end{align*}

\begin{equation}
	L_r=\frac{1.95r_{ts}E}{0.7Fy}\sqrt{\frac{J}{S_x(d-t_f)}+\sqrt{\frac{J}{[S_x(d-t_f)]^2}+6.76\left(\frac{0.7f_y}{E}\right)^2}}
\end{equation}

\FPeval{\lrW}{round(1.95*\rtsW*\Ey/(0.7*\fy)*(\jW/(\sxW*(\dW-\tfW))+(\jW/((\sxW*(\dW-\tfW))^2)+6.76*(0.7*\fy/\Ey)^2)^0.5)^0.5,2)}
\begin{align*}
	 \scriptstyle L_r=\frac{1.95(\rtsW)(\Ey)}{0.7(\fy)}\sqrt{\frac{\jW}{\sxW(\dW-\tfW)}+\sqrt{\frac{\jW}{[\sxW(\dW-\tfW)]^2}+6.76\left(\frac{0.7(\fy)}{\Ey}\right)^2}}\\
	=\lrW\text{ cm}
\end{align*}
%http://akuederle.com/Automatization-with-Latex-and-Python-1
%http://baptisteravina.com/blog/python-to-latex/
\end{document}